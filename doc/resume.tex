\documentclass[margin]{res}
\usepackage[usenames,dvipsnames]{xcolor}
\usepackage[colorlinks=true,pdfstartview=FitV,linkcolor=Black,citecolor=Black,urlcolor=Black,filecolor=Black]{hyperref}
\textwidth=5in

\begin{document}

%% \name{Lang Martin}
%% \address{lang.martin@gmail.com \quad\quad 423 521-0574}

\hbox to \hsize{%
\vbox{\raggedleft
{\bf Lang Martin}\\
lang.martin@gmail.com
423 521-0574\\
4412 St. Elmo Ave., Chattanooga TN 37409\\
\href{https://github.com/langmartin}{GitHub},
\href{www.linkedin.com/in/langmartin2k}{LinkedIn}%
}}

\begin{resume}
\section{About}

I am a software engineer and an engineering manager. I love working on
open, collaborative, and experimental teams, and helping build them.
I'm interested in programming languages, distributed systems,
difficult problems, elegant solutions, and a holistic approach to
system design.

\section{Subspace}
{\bf Distributed Applications Lead}
%
Worked on the control plane systems for configuring the network, lead
the design and implementation effort for the API, and was promoted to
lead of the control plane team. Learned about managing people in a
fully remote setting. Worked on design changes to the core pathfinding
tools, contributing deterministic simulation testing and the
pathfinding algorithms.
%
Sep 2020--

\section{Hashicorp}
{\bf Software Engineer}
%
Worked on the Nomad cluster orchestration program. Developed remote
working skills, planning and communicating clearly in a fully remote
setting. Contributed core features, volunteered on the Raft library
maintenance team, and contributed exploratory code for the Consul
project, the hyparview library.\\
%
Mar 2019--Sep 2020

\section{Half Adder}
{\bf Founder}
%
Built an end user programmable project management system, based on a
CRDT event driven protocol able to use clients as nodes of the
database. Built the libraries, designed the interface, and implemented
the product. Worked with users and business partners to launch the
product.\\
%
Oct 2016--Mar 2019

\section{\href{https://opentable.com/}{OpenTable}}
{\bf Principal Software Engineer}
%
Led a team of five software engineers supporting the core restaurant
operation software. Designed and built a synchronization proxy service
providing deterministic asynchronous data composition for 10k iPad
clients using CRDTs. Designed and built an SMS messaging service used
to notify diners in thousands of restaurants. For both systems,
managed the team that maintained the production deployment.\\
%
Dec 2013--Oct 2016

\section{\href{https://quickcue.com/}{Quickcue}}
{\bf Architect, Lead API Software Enginner}
%
Responsible for the design and implementation of the API, and the
deployment strategy. As a member of the initial team, also involved in
the design of user interfaces. The Quickcue API composed data changes
using operational transform to support offline availability. Quickcue
was acquired by OpenTable.\\
%
Dec 2011--Dec 2013

\section{\href{https://www.cx.com/}{CX}}
{\bf Software Engineer}
%
CX is a Dropbox competitor; worked on the desktop syncing client. The
client featured an embedded Python interpreter that managed the API
and file system, with native wrappers for Windows and macOS.\\
%
Oct 2011--Dec 2011

\section{\href{http://tva.gov/}{TVA}}
{\bf Internal Web Development}
%
Responsible for the specification, design, implementation, deployment
and support of three medium-sized web interfaces to commercial
database applications. Rebuilt a mainframe application to use a web
interface and modern storage layer.\\
%
Apr 2009--Oct 2011

\section{Coptix}
{\bf Programming and Devops}
%
Worked in system administration and later responsible for the
deployment platform. Web development, ASP projects, and leading a
shift to PHP. Instrumental in building a culture of experimentation.\\
%
Sep 1999--Apr 2009

\section{Technologies}

Elixir, Go, Clojure \& ClojureScript, CRDTs \& CQRS, Linux,
PostgreSQL, JavaScript \& TypeScript, Node.js, Scheme

\section{Education}

Covenant College, graduated spring of 2000. B.A.\ in history with
concentrations in education and mathematics.

\end{resume}
\end{document}
